\chapter{Was sind Entscheidungsbäume?}
\label{eb:was-sind-entscheidungsbaeume}



\section{Motivation und Ziel}
\label{eb:motivation}
Das Ziel eines Entscheidungsbaumes ist eine Menge von Regeln aufzustellen anhand derer Datenobjekte automatisch klassifiziert werden können. \autocite{Entschei47:online} 

\section{Generischer Aufbau}
\label{eb:aufbau}
Ein Entscheidungsbaum ist ein geordneter und gerichteter \autocite{Entschei47:online} Baum welcher aus Knoten und Kanten besteht. Dabei unterteilt man Konten in zwei verschiedene Typen. Zum einen gibt es die Entscheidungsknoten \autocites{Decision5:online}{FramworkForSensitivity:online}. Dabei handelt es sich um eine Überprüfung einer Eigenschaft eines Datenobjektes \autocite{DataMining} und letztenendes um eine logische Regel \autocite{Entschei47:online}. Dabei gibt es den Wurzelknoten als besondere Ausprägung ein Entscheidungsknoten. Er unterscheidet sich von einem normalen Entscheidungsknoten nur in soweit, als dass der Wurzelknoten der Ursprung des gesammten Entscheidungsbaumes ist.\\
Zum anderen gibt es die Endknoten