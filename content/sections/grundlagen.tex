\chapter{Was sind Entscheidungsbäume?}
\label{eb:was-sind-entscheidungsbaeume}

Bei Entscheidungsbäumen handelt es sich um eine bestimmte Form von Klassifikationsalgorithmen. 

\section{Motivation und Ziel}
\label{eb:motivation}

\section{Generischer Aufbau}
\label{eb:aufbau}
Im wesentlichen bestehen Entscheidungsbäume aus Knoten und Kanten. Bei einem Knoten handelt es sich um ein zu prüfendes Attribut während es sich bei einer Kante um das Ergebnis dieser Überprüfung handelt. \autocite{DataMining} Darüber hinaus können Knoten wiederrum in Entscheidungsknoten, Wahrscheinlichkeitsknoten und Endknoten unterteilt werden.

Entscheidungsbäume bestehen im wesentlichen aus den vier Bestandteilen Wurzel, Knoten, Kante und Blatt. Bei der Wurzel handelt es sich im Grunde um einen Knoten. 
Bei einem Blatt handelt es sich um eine

\section{Vorteile}
\label{eb:vorteile}

\section{Nachteile}
\label{eb:nachteile}