\chapter{Einführung}
\label{einfuehrung}

Die Frage nach der korrekten Entscheidung bzw. die Herausforderung eine korrekte Vorhersage für ein Ereignis zu treffen, beschäftigt sowohl Ökonomen als auch Wissenschaftler. Zum Beispiel stellt sich die Frage, ob der Kurs eines bestimmten Wertpapiers in der nächsten Zeit steigen oder fallen wird oder ob eine bestimmte Behandlung für einen Patienten empfehlenswert ist. \autocite{QuinlanDecisionTrees}
Zu diesem Zweck kommt unter anderem künstliche Intelligenz zum Einsatz, konkret kommt hierbei besonders maschinelles Lernen im Form der Klassifikation zum Einsatz. \autocite{QuinlanID3} Dabei wird versucht eine Vorhersage anhand von vergangenen Erfahrungen bzw. Daten vorzunehmen, es wird also eine Größe unter Berücksichtigung von verschiedenen Variablen vorhergesagt. Dafür können verschiedene Methoden zum Einsatz kommen, wobei Entscheidungsbäume eine der bekanntesten Methoden ist. Daher werden in dieser Arbeit Entscheidungsbäume näher beleuchtet.